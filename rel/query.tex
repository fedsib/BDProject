\chapter{Query} 
L'interfaccia web utilizza varie query di servizio, di seguito vengono proposte alcune tra le piu' significative adattate a casi particolari.

\section{Query 1}

Recupera la lista di tutti i corsi con i nomi degli istruttori associati.\\

\lstinputlisting[language=SQL,firstline=1,lastline=3]{sql/Query.sql}

\section{Query 2}

Recupero le informazioni riguardanti il corso 4.\\

\lstinputlisting[language=SQL,firstline=5,lastline=9]{sql/Query.sql}

\section{Query 3}

Estrae le prenotazioni fatte dopo la data del 25 agosto 2015, mostrando per ognuna i dati dell'utente e della prenotazione.\\

\lstinputlisting[language=SQL,firstline=11,lastline=15]{sql/Query.sql}

\section{Query 4}
Restituisce, se c'e', una prenotazione per il campo 3 nella data 16-09-2015 alle ore 13.\\

\lstinputlisting[language=SQL,firstline=17,lastline=20]{sql/Query.sql}

\section{Query 5}
Recupero la lista delle lezioni del corso 2 e, se disponibili  anche le prenotazioni corrispondenti.\\

\lstinputlisting[language=SQL,firstline=22,lastline=26]{sql/Query.sql}

\section{Query 6}

Estrae le informazioni di tutti i corsi attivi ai quali l'utente e' iscritto, con nome e cognome del rispettivo istruttore.\\

\lstinputlisting[language=SQL,firstline=28,lastline=37]{sql/Query.sql}

\section{Query 7}

Mostra le prenotazioni dell'utente che visualizza la pagina alla data indicata.\\

\lstinputlisting[language=SQL,firstline=38,lastline=44]{sql/Query.sql}

\chapter{Trigger e Funzioni} 

\section{Trigger}

\subsection{InserimentoRetribuzione}
\lstinputlisting[language=SQL,firstline=98,lastline=110]{sql/DDL_Database.sql}

Il trigger esprime il vincolo semantico che la retribuzione di un'istruttore non possa essere un valore negativo (minore o uguale a zero). Nel caso lo sia, imposta una retribuzione base di 800 euro.
Questo trigger è dedicato alle operazioni di inserimento.

\subsection{AggiornaRetribuzione}
\lstinputlisting[language=SQL,firstline=112,lastline=124]{sql/DDL_Database.sql}

Trigger simile al precedente, utilizzato in caso di aggiornamento del campo retribuzione. Nel caso la retribuzione dell'istruttore venga impostata ad un valore negativo, viene ripristinato il vecchio valore.

Sono stati necessari due trigger visto che MySQL non permette di definire trigger su condizioni multiple.

\subsection{CorsoAttivoIns}
\lstinputlisting[language=SQL,firstline=126,lastline=137]{sql/DDL_Database.sql}

Imposta un corso come non attivo se manca il codice fiscale dell'istruttore che lo tiene. Vale per l'inserimento di nuovi corsi.

\subsection{CorsoAttivoUpd}

\lstinputlisting[language=SQL,firstline=139,lastline=150]{sql/DDL_Database.sql}

Imposta un corso come non attivo se manca il codice fiscale dell'istruttore che lo tiene. Vale per l'aggiornamento dei corsi.

\subsection{CorsoAttivoElim}

\lstinputlisting[language=SQL,firstline=152,lastline=166]{sql/DDL_Database.sql}

Come i trigger precedenti, si occupa di impostare il campo \textit{Attivo} di un corso a false (0) se viene eliminato l'istruttore che tiene quel corso.

\section{Funzioni}

\subsection{ControlloPrenotazione}

\lstinputlisting[language=SQL,firstline=173,lastline=193]{sql/DDL_Database.sql}

\subsection{ControlloPrenotazioneCorso}

\lstinputlisting[language=SQL,firstline=196,lastline=221]{sql/DDL_Database.sql}

\subsection{PossoIscrivermi}

\lstinputlisting[language=SQL,firstline=224,lastline=266]{sql/DDL_Database.sql}