\chapter{Interfaccia Web} 
 
\section{Descrizione interfaccia}

Per interagire con il database e' stata sviluppata un'interfaccia web.

\subsection{Layout}
Si e' optato per un classico layout cosi' strutturato:

\begin{itemize}
\item Spazio per titolo e logo del sito
\item Una barra informativa che indica la posizione dell'utente all'interno del sito e alla destra un link alla pagina di login.
\item Un menu' di navigazione laterale personalizzato in base ai permessi dell'utente che effettua il login.
\item Un box centrale con il contenuto della pagina.
\end{itemize}

\subsection{CSS}

Gli aspetti grafici basilari per il rendering delle pagine web vengono gestiti tramite CSS, particolare attenzione e' stata dedicata alla scelta dei colori per permettere una navigazione del sito anche a persone con problemi di vista.

\section{Descrizione delle pagine}

\section{Funzioni}
Il file \textit{phpfunctions.php} contiene le funzioni base utilizzate da tutte le altre pagine php.

\subsection{connessione}
Effettua la connessione al server.

\subsection{chiusura}
Chiude la connessione al server, quando necessario.

\subsection{loginlink}
Gestisce i link per il login della barra informativa orizzontale distinguendo se l'utente e' loggato o meno.

\subsection{menu}

Genera il menu' verticale laterale.

\subsection{addadmin.php}

Aggiunge un nuovo istruttore che avra' privegi amministrativi.

\subsection{adduser.php}

Aggiunge un nuovo utente con privilegi di default.

\subsection{corsi.php}

Pagina che usa un socio per visualizzare i corsi ai quali e' iscritto e accedere alle informazioni dei corsi.

\subsection{gcorsi.php}

Pagina che usa l'amminsitratore per visualizzare tutti i corsi, aggiungerne di nuovi e accedere alle informazioni dei corsi.

\subsection{gestiscicorso.php}

Visualizza tutte le informazioni del corso, dando accesso alle pagine per vedere gli iscritti, aggiungere le nuove prenotazioni ai campi per le lezioni del corso, modificare le informazioni del corso o cancellarlo. Solo lato admin.

\subsection{gprenotazioni.php}

Permette agli istruttori di vedere, fare ricerche e cancellare prenotazioni.

\subsection{gutenti.php}

Gestione degli utenti, permette agli istruttori di vedere e fare ricerche sugli utenti. Accedere ai dettagli degli utenti ed aggiungere un nuovo account utente o amministratore. 

\subsection{index.php}

La pagina principale del corso.

\subsection{informazionicorso.php}

Permette all'utente di visualizzare informazioni dettagliate di un corso (tra le quali anche le lezioni prenotate), iscriversi in base al suo livello e cancellare l'iscrizione ai corsi.

\subsection{iscritticorso.php}

Visualizza le persone iscritte ad un corso e permette agli istruttori la cancellazione delle loro iscrizioni.

\subsection{login.php}

Gestisce il login.

\subsection{logout.php}

Effettua il logout e rimanda all'index.

\subsection{modadmin.php}

Visualizza informazioni sugli istruttori e permette di modificarle agli istruttori.

\subsection{modificacorso.php}

Permette agli istruttori di modificare le informazioni dei corsi.

\subsection{moduser.php}

Permette agli istruttori di modificare le informazioni dei soci.

\subsection{prenotalezione.php}

Permette di prenotare un campo per una nuova lezione di un corso.

\subsection{prenotazione.php}

In base se l'utente e' socio oppure istruttore permette:
\begin{itemize}
\item A soci e istruttori di vedere le proprie prenotazioni e le prossime lezioni dei corsi.
\item A soci e istruttori di effettuare e rimuovere prenotazioni personali. 
\item Ai soci, vedere le prenotazioni ai corsi ai quali e' iscritto
\item Agli istruttori, di vedere le prenotazioni ai campi delle prossime lezioni dei corsi tenuti.
\end{itemize}

\subsection{utenti.php}

Permette di vedere e modificare le proprie informazioni (tranne la data di assunzione/iscrizione) e cancellare l'account.